\documentclass[a4paper,11pt]{article}

\usepackage[utf8]{inputenc} \usepackage[T1]{fontenc}
\usepackage{fancyhdr} \usepackage{graphicx,subfig} \usepackage{lastpage}
\usepackage{amssymb,amsmath} \usepackage{siunitx} \usepackage[nodayofweek]{datetime}
\usepackage[top=3.5cm,bottom=2.5cm,left=3cm,right=3cm,headheight=40pt]{geometry}
\usepackage{parskip} \usepackage{float} \usepackage{enumitem} \pagestyle{fancy}
\usepackage[colorlinks=true,allcolors=blue]{hyperref} \hypersetup{
	pdfauthor={Michaël Defferrard},
	pdftitle={Project stage 2: Rendering},
	pdfsubject={Introduction to Computer Graphics}
}

\lhead{Introduction to Computer Graphics\\Project stage 2: Rendering\\Group 19}
\chead{\hspace{2.5cm}EPFL\\\hspace{2.5cm}\shortdate\today\\\hspace{2.5cm}\thepage/\pageref{LastPage}}
\rhead{Michaël \textsc{Defferrard}\\Pierre \textsc{Fechting}\\Vu Hiep \textsc{Doan}}
\cfoot{}

\begin{document}


\section{Overview}

This report presents our advancement on the second part of the project : rendering using procedural methods. Figure~ shows an example of what our actual code base is able to generate. All the minimal steps to display a procedurally generated terrain were successfully completed. We did also implement some other optional suggested methods, like 


\section{Implementation}

\subsection{Basic}

\subsubsection{Texturing}

\subsubsection{Modeling the sky}

\subsubsection{Self shadowing}


\subsection{Advanced}

\subsubsection{Approximating water reflections/refractions}

\subsubsection{Water depth effect (participating media)}

\subsubsection{Water dynamics}

\subsubsection{Time of the day}


\section{Results}


\end{document}
